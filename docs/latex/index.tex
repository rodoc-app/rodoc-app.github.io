\chapter{Rodoc}
\hypertarget{index}{}\label{index}\index{Rodoc@{Rodoc}}
\label{index_md_index}%
\Hypertarget{index_md_index}%
 This is the main page for documentation of the program Rodoc.

The code is separated mainly into two parts\+:


\begin{DoxyEnumerate}
\item Core
\item Graphics
\end{DoxyEnumerate}

Core is for working with the family tree and also for parsing files.

Graphics is just the graphical interface of the application.

This application uses \href{https://open-source-parsers.github.io/jsoncpp-docs/doxygen/}{\texttt{ Jsoncpp library}}, \href{https://www.qt.io/}{\texttt{ Qt graphics library}} with version 6.

Database is a group of {\ttfamily json} files and directories. It follows this predefined structure ({\ttfamily \#} is for a comment)\+:


\begin{DoxyCode}{0}
\DoxyCodeLine{[rootDir]\ \ \ \ \ \ \ \ \ \ \#\ Chosen\ by\ the\ user.}
\DoxyCodeLine{\ \ -\/\ Persons.json\ \ \ \#\ All\ persons.}
\DoxyCodeLine{\ \ -\/\ Config.json\ \ \ \ \#\ Configuration\ and\ templates.}
\DoxyCodeLine{\ \ -\/\ Events.json\ \ \ \ \#\ Events.}
\DoxyCodeLine{\ \ -\/\ Relations.json\ \#\ Relations}
\DoxyCodeLine{\ \ -\/\ Files.json\ \ \ \ \ \#\ Files\ metadata.}
\DoxyCodeLine{\ \ -\/\ Media.json\ \ \ \ \ \#\ Media\ metadata.}
\DoxyCodeLine{\ \ -\/\ Notes.json\ \ \ \ \ \#\ Notes\ metadata.}
\DoxyCodeLine{\ \ -\/\ Media\ \ \ \ \ \ \ \ \ \ \#\ Directory\ for\ media.}
\DoxyCodeLine{\ \ \ \ -\/\ [files]}
\DoxyCodeLine{\ \ -\/\ Notes\ \ \ \ \ \ \ \ \ \ \#\ Directory\ for\ notes.}
\DoxyCodeLine{\ \ \ \ -\/\ [file]}
\DoxyCodeLine{\ \ -\/\ Files\ \ \ \ \ \ \ \ \ \ \#\ Directory\ for\ other\ files.}
\DoxyCodeLine{\ \ \ \ -\/\ [files]}

\end{DoxyCode}


Also {\ttfamily Files}, {\ttfamily Notes} and {\ttfamily Media} are internally implemented the same way. The main difference is for the end user. 